%-----------------------------------------------------------------------------------------------------------------------------------------------%
%	The MIT License (MIT)
%
%	Copyright (c) 2021 Jitin Nair
%
%	Permission is hereby granted, free of charge, to any person obtaining a copy
%	of this software and associated documentation files (the "Software"), to deal
%	in the Software without restriction, including without limitation the rights
%	to use, copy, modify, merge, publish, distribute, sublicense, and/or sell
%	copies of the Software, and to permit persons to whom the Software is
%	furnished to do so, subject to the following conditions:
%	
%	THE SOFTWARE IS PROVIDED "AS IS", WITHOUT WARRANTY OF ANY KIND, EXPRESS OR
%	IMPLIED, INCLUDING BUT NOT LIMITED TO THE WARRANTIES OF MERCHANTABILITY,
%	FITNESS FOR A PARTICULAR PURPOSE AND NONINFRINGEMENT. IN NO EVENT SHALL THE
%	AUTHORS OR COPYRIGHT HOLDERS BE LIABLE FOR ANY CLAIM, DAMAGES OR OTHER
%	LIABILITY, WHETHER IN AN ACTION OF CONTRACT, TORT OR OTHERWISE, ARISING FROM,
%	OUT OF OR IN CONNECTION WITH THE SOFTWARE OR THE USE OR OTHER DEALINGS IN
%	THE SOFTWARE.
%	
%
%-----------------------------------------------------------------------------------------------------------------------------------------------%

%----------------------------------------------------------------------------------------
%	DOCUMENT DEFINITION
%----------------------------------------------------------------------------------------

% article class because we want to fully customize the page and not use a cv template
\documentclass[a4paper,12pt]{article}

%----------------------------------------------------------------------------------------
%	FONT
%----------------------------------------------------------------------------------------

% % fontspec allows you to use TTF/OTF fonts directly
% \usepackage{fontspec}
% \defaultfontfeatures{Ligatures=TeX}

% % modified for ShareLaTeX use
% \setmainfont[
% SmallCapsFont = Fontin-SmallCaps.otf,
% BoldFont = Fontin-Bold.otf,
% ItalicFont = Fontin-Italic.otf
% ]
% {Fontin.otf}

%----------------------------------------------------------------------------------------
%	PACKAGES
%----------------------------------------------------------------------------------------
\usepackage{url}
\usepackage{parskip} 	

%other packages for formatting
\RequirePackage{color}
\RequirePackage{graphicx}
\usepackage[usenames,dvipsnames]{xcolor}
\usepackage[scale=0.9]{geometry}

%tabularx environment
\usepackage{tabularx}

%for lists within experience section
\usepackage{enumitem}

% centered version of 'X' col. type
\newcolumntype{C}{>{\centering\arraybackslash}X} 

%to prevent spillover of tabular into next pages
\usepackage{supertabular}
\usepackage{tabularx}
\newlength{\fullcollw}
\setlength{\fullcollw}{0.47\textwidth}

%custom \section
\usepackage{titlesec}				
\usepackage{multicol}
\usepackage{multirow}

%CV Sections inspired by: 
%http://stefano.italians.nl/archives/26
\titleformat{\section}{\Large\scshape\raggedright}{}{0em}{}[\titlerule]
\titlespacing{\section}{0pt}{10pt}{10pt}

%for publications
\usepackage[style=authoryear,sorting=ynt, maxbibnames=2]{biblatex}

%Setup hyperref package, and colours for links
\usepackage[unicode, draft=false]{hyperref}
\definecolor{linkcolour}{rgb}{0,0.2,0.6}
\hypersetup{colorlinks,breaklinks,urlcolor=linkcolour,linkcolor=linkcolour}
\addbibresource{citations.bib}
\setlength\bibitemsep{1em}

%for social icons
\usepackage{fontawesome5}


%%% Работа с русским языком
\usepackage{cmap}					% поиск в PDF
\usepackage{mathtext} 				% русские буквы в формулах
\usepackage[T2A]{fontenc}			% кодировка
\usepackage[utf8]{inputenc}			% кодировка исходного текста
\usepackage[english,russian]{babel}	% локализация и переносы
\usepackage{indentfirst}
\frenchspacing


%debug page outer frames
%\usepackage{showframe}

%----------------------------------------------------------------------------------------
%	BEGIN DOCUMENT
%----------------------------------------------------------------------------------------
\begin{document}

% non-numbered pages
\pagestyle{empty} 

%----------------------------------------------------------------------------------------
%	TITLE
%----------------------------------------------------------------------------------------

% \begin{tabularx}{\linewidth}{ @{}X X@{} }
% \huge{Your Name}\vspace{2pt} & \hfill \emoji{incoming-envelope} email@email.com \\
% \raisebox{-0.05\height}\faGithub\ username \ | \
% \raisebox{-0.00\height}\faLinkedin\ username \ | \ \raisebox{-0.05\height}\faGlobe \ mysite.com  & \hfill \emoji{calling} number
% \end{tabularx}

\begin{tabularx}{\linewidth}{@{} C @{}}
\Huge{Яковлев Дмитрий Алексеевич} \\[7.5pt]
\href{https://github.com/d-a-yakovlev}{\raisebox{-0.05\height}\faGithub\ d-a-yakovlev} \ $|$ \ 
\href{https://t.me/jacob_dmitriev}{\raisebox{-0.05\height}\faTelegram\ @jacob\_dmitriev} \ $|$ \ 
% \href{https://mysite.com}{\raisebox{-0.05\height}\faGlobe \ mysite.com} \ $|$ \ 
\href{mailto:raidegdp@gmail.com}{\raisebox{-0.05\height}\faEnvelope \ raidegdp@gmail.com} \ $|$ \ 
\href{tel:+79084824371}{\raisebox{-0.05\height}\faMobile \ +7(908)482-43-71} \\
\end{tabularx}

%----------------------------------------------------------------------------------------
% EXPERIENCE SECTIONS
%----------------------------------------------------------------------------------------

%Interests/ Keywords/ Summary
\section{О себе}
% This CV can also be automatically complied and published using GitHub Actions. For details, \href{https://github.com/jitinnair1/autoCV}{click here}.
Заканчиваю магистратуру НИУ ВШЭ. Принял для себя решение работать в сфере связанной с обработкой данных, будь это больше аналитическая, инженерная или посвящённая созданию моделей машинного обучения направленность. Поэтому взял перерыв с работой, чтобы сконцентрироваться на учёбе и лучше погрузиться в интересующую область.



%	SKILLS

\section{Навыки}
\begin{tabularx}{\linewidth}{@{}l X@{}}
Языки программирования &  \normalsize{Python, SQL, C/C++}\\
Data analysis  &  \normalsize{numpy, pandas, scipy, matplotlib, seaborn, A/B tests}\\ 
Data science  &  \normalsize{scikit-learn, networkx, CV, NLP, natasha, PyTorch, Tensorflow}\\ 
Data engineering  &  \normalsize{Spark, PostgreSQL, HDFS, MapReduce, RabbitMQ, REST API, Flask}\\  

\end{tabularx}


%	EDUCATION

\section{Образование}
\begin{tabularx}{\linewidth}{@{}l X@{}}	
2022 - 2024 & М \textbf{НИУ ВШЭ} - Системный анализ и математические технологии \hfill \normalsize (GPA: 8.8/10.0) \\

2018 - 2022 & Б  \textbf{ НИУ МИЭТ} - Технические средства автоматизации и управления \hfill \normalsize (GPA: 4.8/5.0) \\ 

\end{tabularx}


%Projects
\section{Проекты}

\begin{tabularx}{\linewidth}{ @{}l r@{} }
\textbf{Проект-дисциплина: "Система реидентификации человека по походке"} & \hfill \href{https://drive.google.com/file/d/1IwEmdHyyXWQlCG_Hfhxn1D-F65usrVAS/view?usp=sharing}{Демо} \\[3.75pt]
\multicolumn{2}{@{}X@{}}{
Проект посвящён созданию прототипного решения для задач видеоналитики данных с камер видеонаблюдения. В рамках проекта занимался созданием моделей и проверкой гипотез. Текущее решение состоит из модели формирующей вектора-походки и индекса в виде KNN для ре-идентификации. Используется оптический поток алгоритма Фарнебака. Stack: Python, PySide/PyQT, OpenCV, PyTorch.
}  \\
\end{tabularx}


\begin{tabularx}{\linewidth}{ @{}l r@{} }
\textbf{Проект Инженерно Математической Школы: "Muse"} & \hfill \href{https://github.com/deepvk/muse}{Ссылка на репозиторий} \\[3.75pt]
\multicolumn{2}{@{}X@{}}{
Проект посвящён задаче разделения сигнала на источники. Приходящая на вход музыка разделяется на: барабаны, басы, вокал и всё оставшееся. В рамках проекта участвовал в оптимизации модели, организации окружения, написании CLI скриптов для работы с решением, формулировании и проверки гипотез. Stack: Python, Docker, PyTorch, TorchAudio, Tensorflow.
}  \\
\end{tabularx}

\begin{tabularx}{\linewidth}{ @{}l r@{} }
\textbf{Проект в рамках курса: "Weather report"} & \hfill \href{https://github.com/d-a-yakovlev/IDA_weather_project}{Ссылка на репозиторий} \\[3.75pt]
\multicolumn{2}{@{}X@{}}{
Проект посвящён предсказанию погоды на ближайшие дни. В рамках проекта организовывал работу команды и писал Flask приложение, объединившее в себе модель, инфографики, frontend и виджет с картой. Stack: Python, SQLite, GeoPandas, Flask, Jinja, HTML, CSS3, JavaScript.
}  \\
\end{tabularx}

%----------------------------------------------------------------------------------------


%----------------------------------------------------------------------------------------
%	PUBLICATIONS
%----------------------------------------------------------------------------------------
% \section{Publications}
% \begin{refsection}[citations.bib]
% \nocite{*}
% \printbibliography[heading=none]
% \end{refsection}

%----------------------------------------------------------------------------------------





%Experience
\section{Опыт работы}

\begin{tabularx}{\linewidth}{ @{}l r@{} }
\textbf{VK} & \hfill Август 2021 - Январь 2023 \\[3.75pt]
\multicolumn{2}{@{}X@{}}{
Начиная как стажёр, впоследствии работал младшим разработчиком. Реализация модулей для обработки различных типов транзакций и добавление новых фич в админку. Stack: Perl, Bash, Docker.
}  \\
\end{tabularx}

\begin{tabularx}{\linewidth}{ @{}l r@{} }
\textbf{Glowbyte Consulting} & \hfill Январь 2021 - Август 2021 \\[3.75pt]
\multicolumn{2}{@{}X@{}}{
Разрабатывал и обновлял ETL для формирования Data Mart. Stack: Spark, Java8, Bash, Hadoop.
}  \\
\end{tabularx}


% \begin{tabularx}{\linewidth}{ @{}l r@{} }
% \textbf{Glowbyte Consulting} & \hfill Январь 2021 - Август 2021 \\[3.75pt]
% \multicolumn{2}{@{}X@{}}{
% \begin{minipage}[t]{\linewidth}
%     \begin{itemize}[nosep,after=\strut, leftmargin=1em, itemsep=3pt]
%         \item[--] Разрабатывал и обновлял ETL, для формировния Data Mart и вспомогательных таблиц.
%         \item[--] Работал с Hadoop кластером: HDFS, Hive.
%     \end{itemize}
%     \end{minipage}
%     Stack: Spark, Java8, Bash, Hadoop.
% }

% \end{tabularx}



\vfill
\center{\footnotesize Последнее обновление: \today}

\end{document}
